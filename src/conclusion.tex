\section{Conclusion}

The entire assignment has been fulfilled successfully. We developed a hardware and software platform that allows the execution of the model predictive controller onboard the micro aerial vehicle. The dynamical model of the helicopter was derived and furthermore its parameters were identified. We proposed a system including the Kalman filter as the state estimator and the quadratic MPC which optimizes control actions over the prediction horizon of 2.2\jed{s}. The proposed system was successfully implemented into the embedded hardware. The controller was verified by simulation and moreover tested by experiments. Many experiments were conducted both indoors and outdoors to test different scenarios inc. tracking various trajectories and disturbance rejection.

This thesis spawned additional work for other students \citep{klucka2015, fiedler2015} --- one aimed to control a group of UAV's synchronously using the XBee modules, another developed a user interface for the ground station and dealt with a failure detection system based on proposed estimator. The platform will continue to be used for  a research of UAV swarms and formations. 

The thesis is an outcome of a long lasting and diligent work. The author is grateful for the knowledge and experience he gained, both theoretical and technical. It is a little but important piece of the puzzle to allow autonomous operation of relatively localized unmanned aircraft. The future is in the sign of pushing frontiers of unmanned vehicles --- let us see what will it bring.