\section{UAV dynamics}

Modeling a system dynamics has always been a great part of a control design process. One with a good understanding of a system's behavior can design a well suited controller using many approaches taught in the field of control theory. Such controller can reflect known characteristics of the system and can appropriately react to evolution of system states. There are two fundamental approaches to the system modeling. One is based on the knowledge about the physics involved in the system. Such knowledge can be used to derive a mathematical model using well known principles, e.g. Hamiltonian mechanics. Control design for such system is usually called \strong{WhiteBox}, or \strong{GreyBox}, depending on how much of the physical process we are able to describe. On the other hand, when the system is unknown, one can create a mathematical model which sufficiently represents observed behavior of the system. Such method is usually called \strong{BlackBox}.

When modeling a vehicle such as classical helicopter, we could create a complex model including phenomena as aerodynamics, rotor-blade flapping and others. In many cases~\citep{alexis2014robust}\citep{mahony2012multirotor}, dynamics of a multirotor MAV can be simplified to a single rigid-body methods. Further due to existence of well designed and tested platforms such as Pixhawk~\citep{pixhawk} or Ardupilot~\citep{ardupilot} we can model the vehicle with the inner feedback loop closed. By doing that, considering a fixed-pitch quadrotor, we move from a system actuated by thrust of four propellers to a system, where the inputs are the desired pitch ($\theta_D$), roll ($\psi_D$), yaw rate ($\dot{\phi}_D$) and collective thrust ($U_D$). It is assumed that such system can be treated as a decoupled system~\citep{mahony2012multirotor}.

\begin{figure}[!h]
\centering
\includegraphics[width=0.7\textwidth]{fig/coordinate_system.pdf}
\caption{UAV and its coordinate systems}
\label{fig:coordinate_system}
\end{figure}

\subsection{Lateral dynamics}

There are several coordinate systems in which we express states of the UAV~(Figure~\ref{fig:coordinate_system}). There is a World coordinate system \textit{W} with a fixed position in the world. Then there is a rotating world coordinate system \textit{R}. It is rotated from $W$ by the angle $\phi$. Following is an inertial frame \textit{I} in which the attitude angles $\theta$ and $\psi$ are measured. It is translated into the geometric center of the UAV. At last there is a Body frame \textit{B} which axis are aligned with the mechanical frame of the UAV.

Assuming a complete decoupling of the system, the lateral dynamics can be expressed by following equations:

\begin{equation}
\begin{split}
\ddot{x}^W &= \frac{U}{m}\left(\sin\psi\cos\phi + \sin\theta\sin\phi\right),\\
\ddot{y}^W &= \frac{U}{m}\left(\sin\theta\cos\phi + \sin\psi\sin\phi\right),\\
\end{split}
\end{equation}

where $U$ is a the total thrust force action on a center of gravity\footnote{in a direction of the $B_z$ axis} and $m$ is the mass of the UAV. Let us assume the desired operating point is a hovering state, where $U$, $m$ are constant and $\theta$, $\psi$ are around zero. We can then simplify the equations into the following form:

\begin{equation}
\begin{split}
\ddot{x}^R &= K_1\sin \phi,\\
\ddot{y}^R &= K_1\sin \theta.\\
\end{split}
\label{eq:attitude_first_lin}
\end{equation}

Since our system is not placed under a global localization system and it relies completely on a dead-reckoning in terms of stabilizing the yaw motion\footnote{The yaw angle $\phi$ is stabilized by a feedback loop implemented on a stabilization board using onboard IMU}, all positions and its derivatives in following equations are expressed in system $R$. Now assuming a low velocity flight around a hovering operating point, these forms can be linearized. It is done by approximating it by first two terms of the Taylor series.

\begin{equation}
\begin{split}
\ddot{x} &= K_1 \phi,\\
\ddot{y} &= K_1 \theta.\\
\end{split}
\end{equation}

\subsection{Altitude dynamics}

The altitude dynamics is described by following equation

\begin{equation}
\begin{split}
\ddot{z} &= \frac{U}{m}\cos\theta\cos\psi - g^W,
\end{split}
\end{equation}

where $g^W$ is a gravitational acceleration. This system is also non-linear, but in this case, we need to be more cautious with a potential linearization. If we try to build an altitude controller which is supposed to work not only around a hovering point, but also during take-off and landing. Unlike in~(\ref{eq:attitude_first_lin}), the force $U$ can not be treated as constant. The pull force of a propeller can be simplified to a quadratic function of its angular speed [\strong{REFERENCE}], which is one of the controlled inputs.

\subsection{Dynamics of integrated stabilization}

Current UAVs are usually equipped with an attitude stabilization system. When building a custom multirotor, such system is a cheap and affordable item on a list. If properly tuned and the feedback loop is closed it transforms the system to be controlled by $\theta_D$, $\psi_D$, $U_D$, $\dot{\phi_D}$ signals. In case of the total thrust transfer, we can assume\footnote{if there is no altitude assisting controller utilizing e.g. barometer, GPS, accelerometer, ...} it takes the form

\begin{equation}
\frac{\mathcal{L}\left\lbrace U \right\rbrace}{\mathcal{L}\left\lbrace U_D \right\rbrace} = 1.
\end{equation}

When properly tuned, the remaining transfers can be fairly modeled using a first order transfer function. Furthermore it is shown that (chapter~\ref{cap:system_identification}) that such model is satisfactory and can be well fitted on a measured data.

\begin{equation}
\begin{split}
\frac{\mathcal{L}\left\lbrace \theta \right\rbrace}{\mathcal{L}\left\lbrace \theta_D \right\rbrace} = \frac{\mathcal{L}\left\lbrace \psi \right\rbrace}{\mathcal{L}\left\lbrace \psi_D \right\rbrace} = \frac{\mathcal{L}\left\lbrace \dot{\phi} \right\rbrace}{\mathcal{L}\left\lbrace \dot{\phi}_D \right\rbrace} = \frac{1}{\tau s + 1},\\
\end{split}
\end{equation}

\subsection{State space representation}

For the purpose of this thesis the discrete formulation of the dynamical system will be used. It allows to formulate proper filtration method and the MPC controller itself. Since now, all differential equations and state space formulation are written in a discrete form with a constant sampling rate. Following form describes a discrete time-invariant system with a main matrix $\mathbf{A}$, and input matrix $\mathbf{B}$

\begin{equation}
\mathbf{q}[n] = \mathbf{A}\mathbf{q}[n-1]+ \mathbf{B}\mathbf{u}[n-1],
\end{equation}

where $\mathbf{x}$[n] is the state vector in the sample $n$. Following matrices describe the pitch and roll systems given the state vectors $\mathbf{q}_{\theta} = \left(x, \dot{x}, \ddot{x}\right)$, $\mathbf{q}_{\psi} = \left(y, \dot{y}, \ddot{y}\right)$ and inputs $\mathbf{u}_\theta = \theta_D$, $\mathbf{u}_\psi = \psi_D$

\begin{equation}
\begin{split}
\mathbf{A}_{\theta, \psi} = \begin{bmatrix}
1 & \Delta t & 0 \\
0 & 1 & \Delta t \\
0 & 0 & P_1
\end{bmatrix}, \mathbf{B}_{\theta, \psi} = \begin{bmatrix}
0\\
0\\
P_2
\end{bmatrix},
\end{split}
\end{equation}

where $\Delta t$ is the sampling period, $P_1$ and $P_2$ are parameters of the first order transfer from a desired to the actual angle of attitude. This system description is a LTI system and can be directly used for state estimation (using e.g. Kalman filter) or for a MPC controller.